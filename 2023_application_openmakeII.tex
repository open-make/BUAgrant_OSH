% Options for packages loaded elsewhere
\PassOptionsToPackage{unicode}{hyperref}
\PassOptionsToPackage{hyphens}{url}
%
\documentclass[
  12pt,
  a4paper,
]{article}
\usepackage{amsmath,amssymb}
\usepackage{lmodern}
\usepackage{setspace}
\usepackage{iftex}
\ifPDFTeX
  \usepackage[T1]{fontenc}
  \usepackage[utf8]{inputenc}
  \usepackage{textcomp} % provide euro and other symbols
\else % if luatex or xetex
  \usepackage{unicode-math}
  \defaultfontfeatures{Scale=MatchLowercase}
  \defaultfontfeatures[\rmfamily]{Ligatures=TeX,Scale=1}
  \setmainfont[]{Times New Roman}
\fi
% Use upquote if available, for straight quotes in verbatim environments
\IfFileExists{upquote.sty}{\usepackage{upquote}}{}
\IfFileExists{microtype.sty}{% use microtype if available
  \usepackage[]{microtype}
  \UseMicrotypeSet[protrusion]{basicmath} % disable protrusion for tt fonts
}{}
\makeatletter
\@ifundefined{KOMAClassName}{% if non-KOMA class
  \IfFileExists{parskip.sty}{%
    \usepackage{parskip}
  }{% else
    \setlength{\parindent}{0pt}
    \setlength{\parskip}{6pt plus 2pt minus 1pt}}
}{% if KOMA class
  \KOMAoptions{parskip=half}}
\makeatother
\usepackage{xcolor}
\IfFileExists{xurl.sty}{\usepackage{xurl}}{} % add URL line breaks if available
\IfFileExists{bookmark.sty}{\usepackage{bookmark}}{\usepackage{hyperref}}
\hypersetup{
  pdftitle={Open.Make II - Implementing FAIR and open hardware},
  hidelinks,
  pdfcreator={LaTeX via pandoc}}
\urlstyle{same} % disable monospaced font for URLs
\usepackage[left = 2cm, right = 1.5cm, top =1cm, bottom =
1.5cm]{geometry}
% Make links footnotes instead of hotlinks:
\DeclareRobustCommand{\href}[2]{#2\footnote{\url{#1}}}
\setlength{\emergencystretch}{3em} % prevent overfull lines
\providecommand{\tightlist}{%
  \setlength{\itemsep}{0pt}\setlength{\parskip}{0pt}}
\setcounter{secnumdepth}{-\maxdimen} % remove section numbering
\ifLuaTeX
  \usepackage{selnolig}  % disable illegal ligatures
\fi

\title{Open.Make II - Implementing FAIR and open hardware}
\usepackage{etoolbox}
\newcommand*{\email}[1]{\href{mailto:#1}{\nolinkurl{#1}} } 
\makeatletter
\providecommand{\subtitle}[1]{% add subtitle to \maketitle
  \apptocmd{\@title}{\par {\large #1 \par}}{}{}
}
\makeatother
\subtitle{30 months project starting in 01/2024}


\author{
Prof.~Dr.-Ing. Roland Jochem \\
Chair of Quality Science, Institute for Machine Tools \\
and Factory Management, Technische Universität Berlin (TUB) \\
\href{mailto:roland.jochem@tu-berlin.de}{\nolinkurl{roland.jochem@tu-berlin.de}}\\
+49 30 314 22005. \\
\and 
Prof.~Matthew Larkum, PhD \\
 \\
Institute of Biology, Humboldt-Universität zu Berlin (HUB). \\
\href{mailto:matthew.larkum@hu-berlin.de}{\nolinkurl{matthew.larkum@hu-berlin.de}}\\
+49 30 450 539 152. \\
\and 
Prof.~Dr.~Tim Landgraf \\
Dahlem Center for Machine Learning and Robotics, \\
Institute of Computer Science, Freie Universität Berlin (FUB) \\
\href{mailto:tim.landgraf@fu-berlin.de}{\nolinkurl{tim.landgraf@fu-berlin.de}}\\
+49 30 838 75114. \\
\and 
Prof.~Dr.~med. Petra Ritter \\
Berlin Institute of Health \& Dept. of Neurology \\
Charité - Universitätsmedizin Berlin (CUB) \\
\href{mailto:petra.ritter@bih-charite.de}{\nolinkurl{petra.ritter@bih-charite.de}}\\
+49 30 450 560 005 \\
\and }

\date{}

\begin{document}
\maketitle

\setstretch{1.15}
\newpage

\hypertarget{introduction-objectives-and-transfer-potential}{%
\section{Introduction, objectives, and transfer
potential}\label{introduction-objectives-and-transfer-potential}}

\hypertarget{summary}{%
\subsection{Summary}\label{summary}}

The open science hardware community has identified a lack of recognition
of research hardware developers and their work as a main bottleneck
towards broader adoption of open hardware in academia. Open.Make II aims
to address this issue by promoting the recognition of (open) research
hardware as a valuable research output. In this project, we will focus
on bridging this gap across various disciplines that involve the use of
hardware, ranging from biology to arts to machine tools. Our vision is
to gradually build an open hardware centre of competence (CoC) that
integrates and connects expertise within~and beyond the BUA. Building on
the insights from the Open.Make project, Open.Make II takes a
socio-technical approach to cultivate a recognised community of
professional open hardware developers and makers in the BUA. While
Open.Make I primarily focused on understanding the requirements for
improving the open hardware space, we will now build on that by
combining, consolidating, and extending the community's tools for
documenting, reviewing, and archiving hardware projects. The emanating
technical foundation for an open hardware publication ecosystem will be
embedded into novel infrastructure of the envisaged CoC: We will provide
support to the community through dedicated spaces, expert advice,
structured educational programs, and extensive networking opportunities.
In conclusion, Open.Make II is poised to make a significant impact by
addressing the recognition gap faced by research hardware developers. By
building an open hardware community within the BUA, we will bring
together diverse expertise, create opportunities for collaboration, and
drive the recognition of research hardware as a valuable research
output. Through our inclusive approach bringing together all four BUA
members and our focus on building robust recognition mechanisms, we aim
to empower open hardware engineers, foster interdisciplinary
collaboration, and advance scientific research on the national and
international level.

\hypertarget{general-context}{%
\subsection{General context}\label{general-context}}

Open access publishing, open data, and free/libre and open source
software (FLOSS) have become important pillars of responsible research
and innovation, an approach for maximising the integrity and impact of
research. The use of the FAIR principles (findable, accessible,
interoperable, and reusable) to promote publication of quality data and
recognition of data sharing as a valuable research task has informed new
research policies. Recent advances of these policies and changes in
mandates from research funders have been successful in promoting greater
adoption of open access to publications and data in science.
Specifically, we see a larger recognition of the work of research
software engineers who publish their work open source. In this context,
the advocacy work of communities of practice (including academics,
citizen scientists, entrepreneurs, artists and teachers) is also pushing
towards the adoption of open hardware in research. Mostly represented by
the Gathering for Open Science Hardware (GOSH)\footnote{\url{https://forum.openhardware.science}}
and the Research Data Alliance interest group ``FAIR principles for
research hardware'' \footnote{\url{https://www.rd-alliance.org/groups/fair-principles-research-hardware}},
these efforts aim to extend open science to include FAIR and open
hardware. Last but not least, open hardware can be seen as a faster and
more sustainable technology transfer route. Protecting intellectual
property is costly and impedes collaborative hardware research and
development. Open hardware can proactively circumvent ``patent
thickets'' in technology areas, fostering a free baseline of knowledge
that enables the emergence of new industries.

Open hardware was first recognised as a pillar of open science
strategies in the UNESCO Open Science recommendation in 2021\footnote{\url{https://en.unesco.org/science-sustainable-future/open-science/recommendation}}.
This global consultation effort harmonised definitions of open science
and provides guidance towards institutionalisation of open science
practices. Researchers worldwide are increasingly producing and sharing
open science hardware designs. Fast-paced research requires highly
customisable instruments and as a result, the interest in better
documenting and sharing hardware designs is growing. Furthermore, FAIR
and open hardware presents an opportunity for new career pathways and an
alternative to traditional intellectual property rights practices,
amplifying the impact of academic research in society.

\hypertarget{central-project-goals}{%
\subsection{Central project goals}\label{central-project-goals}}

This project proposes two central complementary objectives:

\hypertarget{objective-1-foster-a-bua-open-hardware-community}{%
\subsubsection{Objective 1: Foster a BUA open hardware
community}\label{objective-1-foster-a-bua-open-hardware-community}}

Inspired by the TU Delft Open Hardware Community\footnote{\url{https://www.tudelft.nl/en/open-science/about/about/open-hardware}},
this implementation project aims to build an open hardware community
within the BUA context - with the help of numerous actors in Berlin that
are aligned with the project's mission, either relevant in community
building or in hardware production or dissemination. In order to
cultivate an active community, we will provide expertise and
collaboration space, direct mentoring, and networking with both maker
communities inside and outside the universities) and technology transfer
officers at the BUA partners.

In addition, we will consolidate the community by providing tailored
training opportunities by transforming and extending existing materials
into teaching modules and short training formats. These foundational
materials will enable the development of a future curriculum for the
research hardware engineer role.

Ultimately, the team will document the decision criteria, failures and
successes of this work as it takes place. These reflections are aimed to
inform a sustainability strategy to continue the idea of a BUA CoC for
open hardware beyond the duration of the project.

\hypertarget{objective-2-foster-research-hardware-publication-and-recognition}{%
\subsubsection{Objective 2: Foster research hardware publication and
recognition}\label{objective-2-foster-research-hardware-publication-and-recognition}}

Lack of recognition of open hardware work in academia is a
socio-technical problem. In order to facilitate hardware documentation,
publication, and quality control, we will design a software platform
based on the requirements collected via interviews of open hardware
developers and workshops with institutional representatives and
specialists. The team will build and adapt software tooling to
facilitate the edition of the documentation (including generation of
hardware-specific metadata to facilitate discovery), test automatically
for compliance with best practices, create citable and archived
versions, and doing so, add an extra layer of quality control.

An upcoming workshop in the Open Science Conference in June 2023
\footnote{\url{https://www.openmake.de/blog/2023/05/11/2023-05-11-workshop-at-the-open-science-conference/}}
will be an instrumental step in the direction of defining the ecosystem
more broadly and towards deriving a multi-perspective strategy. The team
is seeking collaboration and advice of technical infrastructure
specialists in Berlin and beyond, especially in university libraries.

At the same time, the project team will actively foster the adoption of
FAIR and open hardware. At the local level, we will be liasing with
science funders, university tech transfer offices, and career evaluation
counsellors. Internationally, we will continue the ongoing work inside
the RDA and the GOSH community. This work will allow the recognition of
hardware publication and certification processes internationally, which
is necessary for the adoption of the hardware publication platform.

In addition, we will build new local connections to the Center for the
Science of Materials\footnote{\url{https://www.adlershof.de/en/news/where-the-future-is-materialising}}
via Prof.~Matthew Larkum, that are active in the dissemination of FAIR
and open material. Second, we will investigate relations between open
hardware and medical devices in interaction with the European Testing
and Experimentation Facility Health AI and Robotics
(TEF-Health\footnote{\url{https://www.tefhealth.eu}}) via
Prof.~Dr.~Petra Ritter, and the OpenLab MedTec project\footnote{\url{https://openhsu.ub.hsu-hh.de/bitstream/10.24405/14534/1/openHSU_14534.pdf}}
led by the PTB in Berlin.

\hypertarget{subject-of-implementation-transfer-potential}{%
\subsection{Subject of implementation / transfer
potential}\label{subject-of-implementation-transfer-potential}}

Open.Make II aims to achieve the following outputs: 1) An active and
dynamic BUA open hardware community; 2) open educational resources
(OERs) on open research hardware for practitioners and institutional
stakeholders; 3) a publication platform demonstrator which delivers
first insights from a diverse set of use cases; and 4) community-led
position papers and policy briefs for adoption, for instance on FAIR
principles for research hardware. Community building activities and
teaching formats, combined with an internationally recognised
infrastructure to document and communicate research hardware, will lay
the foundation for a pioneering CoC for open hardware in the BUA. We see
transfer potential as different industries and engineering disciplines
could be inspired to engage in research and technology development
activities based on open hardware principles. We will specifically
analyse the transfer potential for the use cases directly accessible in
our network, such as medical devices, open materials and projects
already in our network (e.g.~the OpenFlexure microscope). Moreover,
developed and refined OERs for training programs for research hardware
developers and makers will be shared to allow replicability of the
approach.

As we will release our self-hostable publication platform and all
associated documentation under open source licenses, open hardware maker
communities outside the traditional academic environment will be able to
reuse and build upon it. We will foster this as part of our continued
outreach activities. A proposal for a national or international
implementation project with higher technology readiness level is
envisaged towards the end of the Open.Make II project to further develop
and exploit the demonstrator.

Moreover, joint policy initiatives will create transfer potentials in
different directions for research hardware. One potential direction is
the very active field of space science (e.g.~NASA Transform to Open
Science (TOPS) programme). Another is the engagement of research funding
agencies such as EC / DFG / BMBF acknowledge research hardware
evaluation practices. Focus groups will be organised on evaluation needs
and how to specify call topic requirements for sharing and collaborative
development of research hardware. The results will be documented and
published as open access policy briefs.

\hypertarget{relation-to-open.make-i}{%
\section{Relation to Open.Make (I)}\label{relation-to-open.make-i}}

Open.Make (I) delivered insights on the following three areas which can
be built upon for further implementation:

\hypertarget{hardware-publication-system-requirements}{%
\subsection{Hardware publication system
requirements}\label{hardware-publication-system-requirements}}

Based on accounts from 15 interviews with representatives of a diverse
set of leading open hardware projects from academia all over the world,
the Open.Make team gathered user stories and has been deriving critical
needs for open hardware development and sharing. The project will
validate these needs in interactive workshops and focus groups with
experts in scholarly communication and infrastructure providers in 2023.
The community will design a roadmap for the creation of a hardware
publication ecosystem. The implementation project will transfer this
knowledge into the creation of a scalable product.

\hypertarget{open-hardware-guidelines}{%
\subsection{Open hardware guidelines}\label{open-hardware-guidelines}}

Taking advantage of our liaison with the Open Hardware Makers training
program\footnote{\url{https://openhardware.space/}} via the incoming
fellowship of Dr.~Julieta Arancio, and with the Turing Way collaborative
community\footnote{\url{https://the-turing-way.netlify.app}}, we will
develop comprehensive guidelines for the development and sharing of open
research hardware in the coming months. This knowledge will be used to
design different training programs inside the BUA, and guides the
development of tools facilitating hardware documentation.

\hypertarget{international-connections}{%
\subsection{International connections}\label{international-connections}}

The project Open.Make has been highly community-oriented with the
foundation of the Research Data Alliance (RDA) „FAIR Principles for
Research Hardware Interest Group`` (endorsed by the RDA in 2022) and the
co-organisation of the global ``unconference'' Gathering for Open
Science Hardware (GOSH) in 2022 in Panama. The Open.Make team has been
tightening its relation with the TU Delft Open Hardware Community and
its managers. Also, the two incoming fellowships (funded by the
Fellowship Programme of Objective 3) of Dr.~Sacha Hodencq from G-SCOP
labs at Université Grenoble Alpes (6 months) and Dr.~Julieta Arancio
from University of Bath (3 months) have been instrumental in solidifying
strong relationships with these institutions, which will be utilised in
Open.Make II. We count on connections to open science hardware
practitioners in the Global South through civil society partnerships
like the Berlin-based association Global Innovation Gathering e.V. (GIG)
and the Latin American chapter of the GOSH community, co-founded by
Dr.~Julietta Arancio. The international network is a prerequisite for
the legitimacy and inclusive nature of of the proposed implementation
solutions in an international context. This network will be strengthened
and new connections will be created.

\hypertarget{positioning-of-the-project-in-relation-to-the-international-state-of-the-art}{%
\section{Positioning of the project in relation to the international
state-of-the-art}\label{positioning-of-the-project-in-relation-to-the-international-state-of-the-art}}

Due to the emerging state of open hardware in academia, we will focus
our efforts on cooperation with existing initiatives, both national and
international, as outlined below:

\hypertarget{open-hardware-institutionalization}{%
\subsection{Open hardware
institutionalization}\label{open-hardware-institutionalization}}

We have a strong connection to the TU Delft which has led the
implementation of open science practices, specifically in terms of open
hardware. The outgoing fellowship gave us a possibility to observe their
work and strategy; we will build on this experiences to adapt the
strategy to the BUA context. TU Delft was the first university to
recently install a position for an open hardware engineer, and build an
open hardware community at the university level. They also developed a
specific curriculum (the Open Hardware Academy), which ran for the first
time in 2022.

\hypertarget{research-hardware-recognition}{%
\subsection{Research hardware
recognition}\label{research-hardware-recognition}}

By founding the FAIR for research hardware RDA interest group, we have
been leading initiatives aiming at the recognition of research hardware
as a research output. There is still much work to do to raise awareness,
but the connection to the RDA community is allowing us to raise the
issues of hardware recognition in international initiatives.

\hypertarget{hardware-publication}{%
\subsection{Hardware publication}\label{hardware-publication}}

Hardware journals exist, but they do not respond to the need of the
community. In particular, the need for a streamlined (publication done
directly from the documentation tool) and free of charge (diamond open
access) system is not well represented in HardwareX, the leading journal
for hardware publication.

Systems for the quality control and dissemination of open hardware have
been emerging during the last years in Germany (OHO - Open Hardware
Observatory e.V., Open Source Ecology Germany e.V.), where the DIN SPEC
3105 was also developed under participation of TUB to define and attest
open hardware. We are closely linked with these different players and
plan to adopt their workflow into our hardware publication ecosystem.

On the other hand, tooling developed for the publication or archival of
software are under development\footnote{\url{https://doi.org/10.48550/arXiv.2201.09015}},
and we are for instance in contact with the HERMES team\footnote{\url{https://project.software-metadata.pub/}}
which is developing tools for the archival of software into data
publication platforms that are often used in university libraries
(Dataverse and InvenioRDM, the latter being the software running
Zenodo). Especially the automatic transfer of specific metadata types is
an interesting and pioneering approach for software publication.

\hypertarget{work-schedule}{%
\section{Work schedule}\label{work-schedule}}

The project's ambitious goals will be tackled as part of four different
types of activities described here as work packages. Actions will be
taken to build a network of open hardware makers in the BUA, create and
provide educational resources, develop a technical platform for hardware
publication and support international initiatives devoted to the
recognition of open hardware as a research output. The work packages
will act synergistically. In \textbf{M24}, we will organise a
mini-conference to present our results and discuss recommendations for
the creation of a sustainable CoC for open hardware in Berlin.

\hypertarget{wp1---maker-community-building-m1---m24}{%
\subsection{WP1 - Maker Community building {[}M1 -
M24{]}}\label{wp1---maker-community-building-m1---m24}}

\textbf{Community initiation and gathering {[}M1 - M7{]}:} We will first
map and connect with existing communities in Berlin and Brandenburg
related to open science (Open Science Working Group of the FU Berlin),
to maker communities inside (workshops / makerspaces, engineers, student
groups) and outside (Top Lab e.V., Motion Lab, Happy Lab, Verbund
offener Werkstätten) the universities, to scholarly communication
(Project HERMES, Open-Access-Büro Berlin), technology transfer, in
addition to the existing BUA community. We will organise community
gathering events as milestone of this task and keep an updated list of
communities on our website.

\textbf{Community building {[}M8 - M24{]}}: Following the Delft open
hardware initiative example, we will provide spaces and mentoring to BUA
hardware developers and makers. A research hardware engineer will help
them create and document their hardware, He will also introduce them to
the guidelines and training offers that will be further developed in
WP2. To mitigate the risk of seeing the rise of a closed community, we
will take particular care in implementing strategies (code of conducts,
easy onboarding) to facilitate the interaction with underrepresented
communities. Concerning the space, we will implement different
strategies. We will have a central workshop {[}from \textbf{M8}{]}, a
pop-up or a mobile workshop {[}from \textbf{M12}{]}, and provide advice
on site {[}from \textbf{M12}{]}. This diversity of activities should
mitigate the risk of reaching a small part of the target group, while
starting with the simpler approach will prevent us to be overwhelmed
with the complexity of the task.

In the last phase of the project, we will target university officers
facilitating the socio-economic impact of academic production and
technology transfer. To this end, we will run an awareness building
campaign that connects open hardware practitioners and university
administration officers in interactive formats. For this, we will join
forces with existing institutions like the Open Hardware Alliance
Germany where TUB is a member. Dedicated sessions {[}in \textbf{M15}{]}
will provide concrete information on how to support and foster the
recognition of open hardware by individual researchers or joint research
projects and how to measure the impact produced by publishing high
quality FAIR and open hardware.

\hypertarget{wp2-education-and-training-m1-m30}{%
\subsection{WP2 -- Education and training {[}M1 --
M30{]}}\label{wp2-education-and-training-m1-m30}}

The first component of WP2 involves aligning and testing already
existing training formats, materials and OERs with the needs of the
emerging community. We will target different status groups, developers
and makers of research hardware from the four BUA partner institutions
considering their respective training needs and expectations. In order
to offer training to early career researchers, students and PIs
interested in open hardware (as part of their research activities), the
team will continuously develop both the project-based university 6 ECTS
seminar with practice partners\footnote{see TUB course:
  \url{https://www.tu.berlin/qw/studium-lehre/lehrveranstaltungen/oshs-open-source-hardware-seminar}}
over three semesters (from summer semester 2024 to summer semester 2025)
and short hybrid training formats (e.g.~1 day) combining theory input
with practice examples and interactive parts. All teaching and training
activities will be held by two team members who will record each others
feedback and create logs for lessons learnt. Participants' expectations
will be surveyed on a voluntary basis before and after interventions
using qualitative as well as quantitative questions for evaluation. The
lessons learnt will be utilised for continuous improvement of the course
module and the training formats on the one hand and to record
reflections of the teaching methodology and learning outcomes in a
scientific paper on the other hand. In addition, developed resources
will be published as OERs that can be taken up by anyone interested and
modified for other target groups.

\hypertarget{wp3---ict-infrastructure-development-m1-m18}{%
\subsection{WP3 - ICT infrastructure development {[}M1 --
M18{]}}\label{wp3---ict-infrastructure-development-m1-m18}}

We will build the technical infrastructure needed to publish hardware in
collaboration with the university libraries and IT infrastructure
specialists that will be invited during another Open.Make workshop in
September/October 2023. The user needs discovered during the first phase
of Open.Make will guide the work and be expanded by external software
development resources as needed. The principal goal for the design of
the ecosystem will be providing support for hardware makers, interfacing
with established practices, and combine existing tools. We have
identified requirements based on interviews and observational studies,
and therefore plan to work on the different aspects of the creation and
publication of research hardware. In the Landgraf lab, the
hardware-related projects, e.g.~a DFG-funded project on robotic fish,
will serve as content to be tested on the publication platform.

\textbf{Documentation creation and automatic checks {[}M1 - M6{]}}: We
will expand the GitBuilding tool, which enables automatic generation of
HTML documentation, adding better onboarding functionalities and
allowing for PDF generation for example. Furthermore we will reuse the
EU-funded software osh-tool, which can be employed in continuous
integration for compliance checks. We will add additional checks for
documentation practices, such as testing for existence and validity of
metadata files or the reusability of source documentation files.

\textbf{Manual checks and peer review {[}M5 - M13{]}}: For the quality
control process, we will reuse and integrate the workflow of the
MediaWiki-based tool developed by OHO and the CADCloud prototype, which
allows for online viewing of CAD files. The scalability of this approach
will be monitored and other types of peer review systems may be built,
tested and implemented.

\textbf{Archival and Recognition {[}M10 - M15{]}}: We will reuse and
adapt workflows developed for software archival by the HERMES project to
export to existing repositories, such as Zenodo. We will also extend
osh-tool to generate Open Badges\footnote{Certificates with embedded
  metadata:
  \url{https://www.imsglobal.org/sites/default/files/Badges/OBv2p0/index.html}}
for attestation of automatic tests. Open Badges will also be test
deployed for attestation of peer review quality.

\textbf{Bundling {[}M12 - 18{]}}: The different components will work
together in a modularised fashion and can be self-hosted. There will be
a web-based control interface facilitating the interoperation of the
individual components. Users will be able to register their git
repositories with it to make use of its services. By only using FLOSS,
we will enable research engineers and libraries to set up their own
instances. The use of metadata standards will allow for the
discoverability of hardware independently of their place of publication.

\hypertarget{wp4---networking-m1---m30}{%
\subsection{WP4 - Networking {[}M1 -
M30{]}}\label{wp4---networking-m1---m30}}

While WP1 - 3 focus on the local/national (BUA, Berlin, Germany)
community, we would like to grow and connect our network
internationally. This will stimulate exchanges of ideas and people, and
will eventually ensure the lasting success of Open.Make II. These
activities are planned to span the entire project period, and will be
led by Mr.~Robert Mies (TUB). We plan to perform several workshops and
present our work in different conferences during the project, especially
RDA plenaries and open science conferences or festivals. We will
continue our efforts inside the RDA group that Dr.~Colomb is presently
co-chairing. A publication of a consolidated declaration document about
the application of FAIR principles for research hardware is planned for
2025.

With the help of the SHK students, we will also consolidate relations
with related projects in Berlin in the materials and medical technology
domains, by means of standardisation efforts for documentation and
dissemination strategies. In particular, we have direct contact to the
newly created Center for the Science of Materials at the HU, as
Prof.~Matthew Larkum is a member of the centre; to the TEF-Health
project, as Prof.~Dr.~Petra Ritter is the coordinator of this consortium
of 51 European partners; and to the OpenLab MedTec project, as the
coordinator PTB in Berlin are our external cooperation partners (see
letter of intent).

\hypertarget{milestone-plan}{%
\subsection{Milestone plan}\label{milestone-plan}}

\textbf{Milestone 1, M6, WP1} -- Establishing of local network /
\textbf{Means of verification (MoV)}: Relevant connections made in
Berlin area including as part of a community gathering event with the
main target groups and internal planning for continuous engagement
formalised. \textbf{Milestone 2, M9, WP2} - Short format training/
\textbf{MoV}: Report with lessons learnt, about the first short format
training workshop for researchers. \textbf{Milestone 3, M13, WP1} -
Mobile workshop / \textbf{MoV}: A mobile or pop up workshop is set to a
first location in Berlin. \textbf{Milestone 4, M13, WP3} -- Technology
validation / \textbf{MoV}: First working prototype of technology
infrastructure ready for testing in the lab. \textbf{Milestone 5, M24,
WP4} - Closing out of RDA initiative / \textbf{MoV} - Submission of RDA
declaration for final review.

\hypertarget{risk-assessment}{%
\subsection{Risk assessment}\label{risk-assessment}}

Adoption of new technology may be harder than it is initially thought
(WP3). This is often due to user interfaces (UI) not matching the user
needs, so surveys will be used to investigate this issue. The planned
hire of external designers to improve the UI will be beneficial as well.
By building on existing tools already in use, we can moreover make sure
that the tools will be able to archive hardware documentation of a
certain quality standard. On the other hand, no manual peer review
system has been tested at scale yet (WP1 \& 3). Indeed, attestation of
quality is a difficult question and we may need to follow different
socio-technological paths, and use different tools. We will therefore
continue discussions with the scholarly communication ecosystem,
research hardware and software engineers, and other interested
communities, in order to find working solutions that address their
needs. Experience gathered inside the Open Hardware Observatory
\footnote{\url{https://en.oho.wiki/wiki/Home}} and Open Source Ecology
Germany \footnote{\url{https://wiki.opensourceecology.de/Open_Source_Ecology_Germany}}
hardware review attempts will be additional resources to build the
platform in a user-friendly way. Additionally, community, teaching and
networking activities (WP1, 2 \& 4) may not reach the target groups, so
dissemination and communication through existing channels used in
Open.Make (1) and through the BUA and partnering institutions will be
used to mitigate this. Finally, should too much interest be created it
can be managed by using the network to buffer spikes. We are confident
that the expertise of the Open.Make team and BUA partners, combined with
external help from the Open.Make network developed in WP4 will overcome
these challenges.

\hypertarget{information-on-potential-practical-use-of-results}{%
\section{Information on potential practical use of
results}\label{information-on-potential-practical-use-of-results}}

Our results will have a practical and replicable nature by design. OERs
are meant to be reused in different training formats and university
courses offered by Open.Make PIs. The environment for hardware
documentation and publication will be based on FLOSS and streamline the
documentation, reviewing, and archival of hardware design. The tool is
therefore aimed at users both inside and outside academia. The envisaged
Berlin hardware developer and maker community will help each of its
members to thrive for producing quality open hardware, following the
FAIR principles. This will widen the reach of every project, and we will
therefore enable future practical use of these hardware pieces.

\hypertarget{concept-of-implementation-and-dissemination-of-potential-applications}{%
\subsection{Concept of implementation and dissemination of potential
applications}\label{concept-of-implementation-and-dissemination-of-potential-applications}}

We believe this project will build the foundation for the development
and recognition of open hardware in research (training concepts,
guidelines aimed at the different target groups, publication and
recognition system), and serve as an enabler for open hardware
advocates. We will foster this advocacy locally, and internationally,
both by raising awareness and by documenting our processes.

By continuing to interconnect through the BUA, e.g.~through workshops
and events, the project will increase the local visibility of the
project itself, as well as the visibility of other open hardware
research projects in Berlin. To ensure inclusivity and diversity, we
will actively involve the non-academic makers community from the early
stages. Through interactive sessions and networking opportunities, we
will establish a common ground between research hardware developers,
university officers, makers outside academia, citizen scientists and
interested teachers. This approach, proven successful in Delft and
exemplified by the Libre Solar project\footnote{\url{https://libre.solar/}},
which we interviewed during Open.Make, enables cross-pollination of
ideas and experiences. It also allows for the dissemination of our
outputs within civil society, particularly at the local level.

Open.Make (I) has been investigating best practices in research hardware
development and dissemination. While doing so, we have built a strong
network with other actors in this sector, especially through GOSH and
the RDA. We will continue to actively recruit additional collaborators
and grow our network, especially within the new framework of the RDA
group FAIR4RH that we co-founded as part of Open.Make (I). It is
currently co-chaired by Dr.~Nadica Miljković (University of Belgrade)
and Dr.~Julien Colomb and has grown to 45 members. This growing network
will be used to disseminate the applications of the publication platform
and our educational resources. We will support these communities that
are working on introducing policies for funders and institutions on open
research hardware, as having more OERs and software tools to recommend
to practitioners will strengthen their positions. In particular, we will
continue to collaborate with civil society partners like the Open
Hardware Alliance (which is led by the Open Knowledge Foundation) and
the Internet of Production Alliance\footnote{\url{https://www.internetofproduction.org/}}.
The former is doing policy work for the recognition of open hardware as
a viable research output to maximise societal impact and the latter is
developing community standards for open hardware.

By documenting success and failures in our community building tasks, we
will make it possible for other players to follow the same path and
build hardware communities as part of future open source programme
offices.

\hypertarget{exploitation-plan-for-academic-and-non-academic-users}{%
\subsection{Exploitation plan for academic and non-academic
users}\label{exploitation-plan-for-academic-and-non-academic-users}}

We will ensure sustainability by securing long term funding for the
exploitation of our outputs (the hardware platform and the community).
By making it possible for academic and non-academic users to continue
the work, our outputs will rely less on the core team. We will take
special attention to include low and middle income countries as putative
partners in the exploitation of our outputs. In addition, we will use
this program to present the BUA with a long term sustainability plan
involving the creation of a Center of Competence for research hardware
in Berlin. Concerning the hardware publication platform, we will on one
hand engage with institutions and communities that can deploy the tool
at scale. University libraries (also outside of the BUA) will be a
strong partners, especially for the long term archival of hardware
documentation. The GOSH community is another strong partner who may
exploit our tool in the long term. We are also in direct contact with
the Invest in Open Infrastructure \footnote{\url{https://investinopen.org}}community
(via our interaction with the Turing Way community) that might assist us
to expand our platform demonstrator into a sustainable, diamond open
access solution for hardware publication. On the other hand, our work
will be open sourced from the start, and communities around the globe
will be actively encouraged to follow the project's steps. The tools we
will expand can be used independently and there already are communities
feeling responsible for their maintenance. By engaging with them we will
secure long term use of our software outputs.

Concerning our educational resources, the student training will be
included in the normal curriculum of the TU and FU. In addition, the
material can suit a variety of needs, according to different audiences:
from university teaching to research training to school lessons, citizen
science projects or even entrepreneurial activities. We therefore expect
to see open hardware training flourish both inside and outside academia
during this grant period.

The project can also serve as a catalyst for addressing social issues,
particularly in fostering collaborations with the Global South.
Ingenieure ohne Grenzen e.V. (IOG) together with OHO e.V. and TUB
(through Robert Mies) are starting a working group within IOG to explore
the potential benefits of open hardware documentation and sharing for
international development cooperations. Dr.~Julieta Arancio, our
incoming researcher, is actively engaged in the Latin American open
science hardware landscape and is settling in Berlin. The Global
Innovation Gathering (GIG) network, based in Berlin, will play a vital
role in informing the development of Open.Make II. Through their
connections with maker spaces in Africa and South Asia, we aim to
leverage these networks to enhance the platform and training materials,
as well as explore another avenue for future exploitation potential.

\hypertarget{partners-and-target-groups-in-berlin}{%
\subsection{Partners and target groups in
Berlin}\label{partners-and-target-groups-in-berlin}}

As described earlier, significant efforts will be dedicated to search
and connect with local hubs and BUA expertise. In particular, we will
seek a closer relation with university libraries for the development and
implementation of the publication platform. This will start later this
year inside the Open.Make project. Similarly, we will reach to
intellectual property officers, open access and research data management
offices in this platform design phase. We will also network with Berlin
graduate schools to foster the teaching of open hardware (and open
science practices) at the PhD level. We engage in inactive exchanges
with neuroscientist of the CRC1315, which define the publication of FAIR
hardware documentation as one objective of their infrastructure
project\footnote{\url{https://www.sfb1315.de/research/inf/}}.

During Open.Make II, we will collaborate with institute workshops (for
instance the Feinwerktechnik, a workshop service at FUB) inside WP1.
Service facilities like these build a significant part of the hardware
used in research groups. FUB's Feinwerktechnik has agreed to work with
us and implement the Open.Make guidelines in future constructions. We
will find other interested parties inside the four partner
organisations, such as the TU-based workshop of the excellence cluster
``Science of intelligence''. This network will then be leveraged to
foster the adoption of hardware training and publication inside the BUA.
This work will provide the ground knowledge and practical recommendation
for a future implementation of an open hardware strategy in Berlin and
beyond.

\hypertarget{concept-for-collaboration-with-project-partners}{%
\section{Concept for collaboration with project
partners}\label{concept-for-collaboration-with-project-partners}}

Open.Make II will involve all four partner institutions of the BUA. The
team that made Open.Make (I) a success, will be extended by
Prof.~Dr.~Petra Ritter (CUB) for an additional focus point on open
hardware in the health space. We have extensive expertise in hardware
evaluation (TUB), academic OS hardware \& software development (FUB,
CUB), scholarly communication and data management at (HUB), and medical
applications (CUB). While Dr.~Julien Colomb will continue to work
part-time on this project (about 15\%) inside his contract in the
infrastructure project of the SFB1315, both Robert Mies and Moritz
Maxeiner will be dedicating their full attention on open Make II.
Prof.~Tim Landgraf has a number of hardware-related projects that will
serve as example use-cases for the publication platform demonstrator. He
is also mentoring an active startup in the digital health space and is
interested to explore open hardware options with them. Prof.~Dr.~Petra
Ritter is the coordinator of TEF-Health, a consortium of 51 European
groups tasked with validating new testing protocols for AI and robotics
solutions in the healthcare sector. This field is particularly
interesting as regulations on how to document medical hardware
throughout development might be translated to non-medical hardware.
TEF-Health will stimulate the transfer from research to clinical
practise and will be a powerful addition to the Open.Make network.

The team remains committed to achieve the long-term objectives we have
set for ourselves to foster open hardware in academia from Berlin. In
particular, the work planned for WP1 and WP2 will require high levels of
collaboration to ensure the contents of the workshops and training
programs are well-rounded. With regards to individual responsibilities,
the TUB will take the lead in project coordination and overall
organisation of WP1 and WP2, while the HUB will head data management
tasks and international networking (WP4). The FUB and TUB will have
shared responsibilities for cooperation with OH communities with the TUB
focusing on national groups. As mentioned previously, the FUB will
furthermore lead the task of software development (WP3), on top of their
role as research hardware engineer (WP1).

Robert Mies (TUB), Dr.~Julien Colomb (HUB), and Moritz Maxeiner (FUB)
are foreseen as the main resources during the whole grant period. The
current work organisation and data sharing habits will be kept. The TUB
HiDrive will be used for internal data exchange, while public data will
be pushed on GitHub. All partners will share relevant files in the
consortium and prepare their documentation to be released and published.
Where legally possible, we will work in the open, so that putative
partners can follow in progress in real time, see www.openmake.de.

This will explicitly be the case for the publication system software,
which will be open source and developed in a public git repository. For
this, we will use the GitLab development and operations platform, unless
we expand a project that is hosted elsewhere. The platform has been used
successfully and heavily by FUB for software project management,
development, and continuous integration. We believe this open approach
will foster adoption of the system. Bi-weekly team meetings have proven
invaluable in monitoring the progress of the current project and we will
thus keep with that process. We will also continue organising a whole
team meeting at least once every six months, with the agenda being
prepared by TUB.

\hypertarget{research-data-management}{%
\section{Research data management}\label{research-data-management}}

As for Open.Make (I), all outputs of the project will be available with
open access as soon as possible for the community. We will continue to
use our website (www.openmake.de) as a blog platform to share grey
literature. Hardware and software will be build in the open using one or
several Git platform(s) and published (on Zenodo for software) once
ready. The RDA platform will be used to publish our community-created
outreach document, in particular our work on the application of FAIR
principles for hardware. Publications will be gold open access under
Creative Commons license and additionally published in university
long-term archiving repositories, such as DepositOnce (TUB).

\newpage

\hypertarget{financial-plan}{%
\section{Financial plan}\label{financial-plan}}

\begin{itemize}
\tightlist
\item
  Total estimated costs: \textbf{465 012 EUR} (2024: 222 453 EUR + 2025:
  209 002 EUR + 2026: 31 557 EUR)
\item
  Personnel costs: \textbf{413 012 EUR} (2024: 188 953 EUR + 2025: 193
  502 EUR + 2026: 30 557 EUR)
\item
  Other costs \textbf{52 000 EUR} (2024: 34 000 EUR + 2025: 15 500 EUR +
  2026: 2 000 EUR)
\end{itemize}

The budget plan slightly exceeds the maximum amount allowed. We hope
this is acceptable as the objectives of the project are very ambitious
and all partner institutions of the BUA are playing a vital role in
Open.Make II.

\hypertarget{personal-costs}{%
\subsection{Personal costs}\label{personal-costs}}

\begin{itemize}
\tightlist
\item
  TUB: One postdoc 24 months 100\% + 6 months 50\%: \textbf{235 662 EUR}
  (2024\_12 months 100\%: 103 015 EUR + 2025\_12 months 100\%: 105 590
  EUR + 2026\_6 months 50\%: 27 057 EUR)
\item
  FUB: 24 person months for one doctoral researcher 100\%: \textbf{151
  831 EUR} (2024\_12 months: 74 978 EUR + 2025\_12 months: 76 853 EUR)
\item
  HUB: One student assistant 30 months 40 h per month: \textbf{17 500
  EUR} (2024\_12 months: 7 000 EUR + 2025\_12 months: 7 000 EUR +
  2026\_6 months: 3 500 EUR)
\item
  CUB: One student assistant 12 months 40 h per month: \textbf{8 019}
  EUR (2024\_6 months: 3 960 EUR + 2025\_6 months: 4 059 EUR)
\end{itemize}

Note: HUB and CUB, via the SFB1315 INF project, provide a senior post
doc 15\% (Dr.~Julien Colomb) as existing staff for the 30 months.

\hypertarget{publications-travels-conference-fees-events-server-hostingmaintenance}{%
\subsection{Publications, travels, conference fees, events, server
hosting/maintenance}\label{publications-travels-conference-fees-events-server-hostingmaintenance}}

\begin{itemize}
\tightlist
\item
  TUB: \textbf{5 000 EUR} (2024: 2 000 EUR + 2025: 2 000 EUR + 2026: 1
  000 EUR)
\item
  FUB: \textbf{5 000 EUR} (2024: 2 500 EUR + 2025: 2 500 EUR)
\item
  HUB: \textbf{5 000 EUR} (2024: 2 000 EUR + 2025: 2 000 EUR + 2026: 1
  000 EUR)
\item
  CUB: \textbf{2 000 EUR} (2024: 1 000 EUR + 2025: 1 000 EUR)
\end{itemize}

\hypertarget{workshop-tools-and-consumables-wp1}{%
\subsection{Workshop tools and consumables
(WP1)}\label{workshop-tools-and-consumables-wp1}}

TUB: \textbf{20 000 EUR} (2024\_16 000 EUR for 2x machine tools \& 2 000
EUR for materials and consumables: 18 000 EUR + 2025\_materials and
consumables: 2 000 EUR)

\hypertarget{external-services-wp3}{%
\subsection{External services (WP3)}\label{external-services-wp3}}

FUB: \textbf{15 000 EUR} (2024\_software development: 9 000 EUR + 2025\_
web and user experience design: 6 000 EUR)

\end{document}
